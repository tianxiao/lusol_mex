\documentclass[11pt]{article}
\usepackage{geometry}
\geometry{letterpaper}
%\geometry{landscape}
\usepackage[parfill]{parskip} 
\usepackage{graphicx}
\usepackage{amssymb}
\usepackage{amsmath}
\usepackage{epstopdf}
\usepackage{algorithmic}
\usepackage{algorithm}
\usepackage{cite}
\usepackage{xspace}

%% some math commands
\newcommand{\T}{^T}
\newcommand{\lmex}{\texttt{lusol\_mex}\xspace}

\title{LUSOL Matlab Interface Tutorial}
\author{Nick Henderson}
%\date{}

\usepackage[pdftex,colorlinks]{hyperref}

\begin{document}
\maketitle

\section{Introduction}

This tutorial documents the installation and use of a Matlab mex interface to
Michael Saunders'
\href{http://www.stanford.edu/group/SOL/software/lusol.html}{LUSOL}.  The
interface, named \lmex, uses LUSOL to compute a sparse LU factorization of a
matrix.  The interface provides access to all solve, multiply, and update
subroutines.  Source code and binaries can be downloaded from

\begin{center}
\url{https://github.com/nwh/lusol_mex/}
\end{center}

\section{System requirements}

\begin{itemize}
\item Matlab version 7.6 (R2008a) or newer.  \lmex uses a style of
  object-oriented programming that was introduced in R2008a.  Earlier versions
  of Matlab are not supported.
\item Compatible C compiler.  Each version of Matlab is compatible with a
  certain compiler.  Refer to the
  \href{http://www.mathworks.com/support/compilers/previous_releases.html}{list}
  on Mathwork's website.
\item Compatible fortran compiler.  This need not be the compiler specified in
  the Matlab list.  The fortran compiler is used to for code that will be
  linked to the mex routine.  It is required that the fortran compiler be
  ``compatible'' with the C compiler that mex uses.
\end{itemize}

\lmex has been successfully built under a few different scenarios:
\begin{itemize}
\item Mac OS X 10.6, Matlab R2009a, \texttt{gcc} from Xcode, and
  \href{http://www.g95.org/}{\tt g95} or
  \href{http://r.research.att.com/tools/}{\tt gfortran}.
\item Ubuntu 10.10 64-bit, Matlab R2010b, \texttt{gcc-4.3}, and
  \texttt{gfortran-4.3}.
\end{itemize}

\section{Building}

\begin{enumerate}
\item Run \texttt{mex -setup} to generate
  \texttt{\$HOME/.matlab/[VER]/mexopts.sh}.
\item Modify \texttt{\$HOME/.matlab/[VER]/mexopts.sh} to point to the correct
  compiler version.  The change must be made in the section corresponding to
  your system architecture.  For example, variables for 64-bit linux are set in
  the \texttt{glnxa64} section.
\item Modify the \texttt{makefile} in the \texttt{lusol\_mex/} directory.
\item Run \texttt{make}.
\item Make sure the \texttt{lusol\_mex/} directory is added to your matlab
  path.
\end{enumerate}

\section{Testing}

There is a small set of test cases in the \lmex directory.  These require the
Matlab xUnit Test Framework to run.  See:

\begin{center}
\url{http://www.mathworks.com/matlabcentral/fileexchange/22846-matlab-xunit-test-framework}
\end{center}

With the testing framework installed, the command \texttt{runtests} will
execute all tests in \texttt{lusol\_test.m}.

\section{Design}

\lmex makes use of object-oriented programming in Matlab for efficiency and
ease of use.  Here is an example that factorizes a random sparse matrix:

\begin{verbatim}
>> rand('twister',123);
>> A = sprand(30,30,.2);
>> mylu = lusol(A);
\end{verbatim}

The symbol \texttt{mylu} now refers to an object in the Matlab workspace.  It
contains all of the data for the sparse factorization and provides access to
methods for solving $Ax=b$, computing products $Ax$, and performing stable
sparse updates.

\begin{verbatim}
>> % solve Ax=b
>> b = ones(30,1);
>> x = mylu.solve(b);
>>
>> % multiply Ax
>> b2 = mylu.mulA(x);
>>
>> % check the result
>> norm(b-b2)

ans =

   1.1285e-14
\end{verbatim}

Direct access to the data is not provided and usually not needed.  It
is possible to obtain sparse $L$ and $U$ factors with the methods:

\begin{verbatim}
>> L = mylu.L0();
>> U = mylu.U();
\end{verbatim}

The method to obtain the $L$ factor is named \texttt{L0} to indicate
that it returns the initial $L$ factor.  LUSOL stores updated to the
$L$ factor in product form.  The $U$ factor is always maintained.  Due
to the structure of LUSOL data, the \texttt{L0} and \texttt{U} methods
are required to copy and manipulate data.

\section{Usage}

The best source of information on the usage of \lmex is with Matlab's
\texttt{help} and \texttt{doc} commands.  Try:

\begin{verbatim}
>> help lusol
>> % or
>> doc lusol
\end{verbatim}

The first thing to do is to create a \texttt{lusol} object.  This
requires a matrix $A$ and possibly some options:

\begin{verbatim}
>> mylu = lusol(A);
>> mylu = lusol(A,options);
>> mylu = lusol(A,'pivot','TRP','Ltol1','5.0');
\end{verbatim}

The first command instantiates the \texttt{lusol} object and
factorizes $A$.  Matrix $A$ can be scalar (\texttt{A = 1}), however it
may not be empty (\texttt{A = []}).  The second command sets
parameters using the \texttt{options} struct.  The third command sets
parameters using the key-value format.

\subsection{Options}

See \texttt{help lusol.luset} for details.  The best documentation for
the input parameters is the fortran code in the file \texttt{lusol.f}.

To create an options structure with defaults:

\begin{verbatim}
>> options = lusol.luset();
\end{verbatim}

Options may then be changed by modifying the structure fields.  You
can also change the default options using key-value pairs in the
parameter list to \texttt{lusol.luset}.  For example:

\begin{verbatim}
>> options = lusol.luset('pivot','TRP');
\end{verbatim}

Note that \texttt{lusol.luset} is a ``static'' method.  It can be
called without creating a \texttt{lusol} object.

\subsection{Factorize}

The \texttt{factorize} method can be used to factorize a (new) matrix
after a \texttt{lusol} object has already been created.  Reallocation
will only occur if the new matrix requires more storage or if the
parameter \texttt{nzinit} is set larger. Example:

\begin{verbatim}
>> % mylu is already in the workspace
>> [info nsing depcol] = mylu.factorize(A,options);
\end{verbatim}

The output:
\begin{description}
\item[\tt info] a status flag
\item[\tt nsing] number of apparent singularities
\item[\tt depcol] logical index indicating dependent columns
\end{description}

\subsection{Solve}

The \texttt{solve} method allows solves with $A$, $A\T$, $L$, $L\T$,
$U$, $U\T$.  See \texttt{help lusol.solve}.  The relevant methods are:

\begin{itemize}
\item \texttt{solveA}
\item \texttt{solveAt}
\item \texttt{solveL}
\item \texttt{solveLt}
\item \texttt{solveU}
\item \texttt{solveUt}
\end{itemize}

These methods all call \texttt{solve} with the correct mode.

\subsection{Multiply}

The \texttt{mul} method allows products with $A$, $A\T$, $L$, $L\T$,
$U$, $U\T$.  See \texttt{help lusol.mul}.  The relevant methods are:

\begin{itemize}
\item \texttt{mulA}
\item \texttt{mulAt}
\item \texttt{mulL}
\item \texttt{mulLt}
\item \texttt{mulU}
\item \texttt{mulUt}
\end{itemize}

These methods all call \texttt{mul} with the correct mode.

\subsection{Update}

\lmex provides access to all update subroutines in LUSOL:

\begin{description}
\item[\texttt{repcol}] replace a column
\item[\texttt{reprow}] replace a row
\item[\texttt{addcol}] add a row
\item[\texttt{addrow}] add a column
\item[\texttt{delcol}] delete a row
\item[\texttt{delrow}] delete a column
\item[\texttt{r1mod}] rank 1 update
\end{description}

\end{document}